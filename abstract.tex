\begin{center}
    \Large
    \themebold{Clustering algorithms in the touristic sector}
    
    \vspace{0.4cm}
    \large
    Case study on travel agency data
    
    \vspace{0.4cm}
    \themebold{Argyro Sioziou}
    
    \vspace{1.5cm}
    \themebold{Abstract} 
\\
\vspace{25mm}
Machine learning, and especially clustering, are nowadays broadly used by businesses of many different sectors, to extract knowledge from their data, and use it in favor of themselves and their customers. The tourism sector also seeks to exploit its data and gain insights from them. Though, this sector, deals with the problem that its data in its majority are categorical, and therefore harder to analyze. In this thesis, an introduction of machine learning and clustering is made and two algorithms are thoroughly analyzed. Afterward, a case study is presented that shows the importance of clustering techniques for tourism, using the data of a travel agency. The case study proves, that even though categorical data make it harder to extract meaningful results, machine learning still constitutes a very useful way to guide tourism businesses through their decisions. 
\end{center}