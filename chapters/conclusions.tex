\section{Comparing Results}
\section{Restrictions}
Even though the dataset contained a sufficient volume of data, these were not evenly distributed throughout the different kinds of information. The data about the accommodations were severely outnumbering the excursion data. This resulted in the latter not playing any role in the cluster formation. The main reason why the excursion data were insufficient was due to the misuse of the backoffice system. Initially, a large amount of excursion data were available, but not properly inserted, at the time of the reservation, in the system and therefore was not possible to connect them to their corresponding accommodation reservation. \\
Furthermore, as mentioned, the clusters were to a large extent formed by the type of accommodation. If the type of accommodation was to be translated as the benefits that a client seeks on, for example, renting a studio instead of an apartment the results could offer a better picture of the factors that affect a traveler's decisions. This type of information was not currently available.
\section{Future Work}