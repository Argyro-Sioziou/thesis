\section{Comparing Results}
The task of dealing with categorical data is demanding and requires good data manipulation to produce satisfactory results. Through the usage of the two algorithms, we could say that both of them worked pretty well with them and provided useful information. Especially Kmeans, even though it is not considered to work well with this type of data, produced the best results out of two, managing to recognize all the four groups. \\
Indeed the results could have been better if the categorical data did not affect them that much. Still, the travel agency can take advantage of the outcome to adjust both its product and marketing strategy in favor of the clusters 'Luxury trip' and 'Relaxing vacation' to become more competitive in the touristic market and increase its revenue.
\section{Restrictions}
Even though the dataset contained a sufficient volume of data, these were not evenly distributed throughout the different kinds of information. The data about the accommodations were severely outnumbering the excursion data. This resulted in the latter not playing any role in the cluster formation. The main reason why the excursion data were insufficient was due to the misuse of the backoffice system. Initially, a large amount of excursion data were available, but not properly inserted, at the time of the reservation, in the system and therefore was not possible to connect them to their corresponding accommodation reservation. \\
Furthermore, as mentioned, the clusters were to a large extent formed by the type of accommodation. If the type of accommodation was to be translated as the benefits that a client seeks on, for example, renting a studio instead of an apartment the results could offer a better picture of the factors that affect a traveler's decisions. This type of information was not currently available.
\section{Future Work}
The analysis did result in useful insights for the agency. Additional analysis though could be done if the current dataset was enriched by survey data given to the travelers. These data could be about the benefits sought from their trips, and the factors that lead them to choose the services they chose. In this way, some categorical data could be converted to ordinal and the analysis would produce better results. \\
Furthermore, the same way this analysis was mainly focused on the accommodations since they could not be combined with all the equivalent excursions, a second analysis could be conducted based on the excursions. This analysis could produce excursions bundles to be promoted together. Finally, the agency's portfolio of excursions could be expanded based on the most profitable ones.