\section{Research Motivation}
The fourth industrial revolution has come with many new technologies, and even though it has just begun, it is affecting people's lives and companies' behaviors in many aspects \autocite{revolution}. The latter are searching for new ways to address the former and engage them, while their expectations are continuously increasing. This leads companies of many industries, including tourism, to intend to deliver personalized experiences, using tools like marketing \autocite{engagement}. At the same time, the available data volume is increasing rapidly, giving the ability to enterprises worldwide to extract knowledge from them and create value \autocite{bigdata}. \\
This plethora of data provides an opportunity for marketers of the tourism sector, amongst others, to perform better segmentation, and get insights into their customers' needs using cluster analysis \autocite{fuzzyturism}. The results of such techniques can also help businesses of the industry in their decision-making process and the creation of competitive advantage \autocite{advantage}. However, this task is not that simple. Turistic data are to a large extent qualitative, which makes it hard for analysts to extract the maximum knowledge from them since clustering algorithms are in their majority used with quantitive data \autocite{categorical}. \\
The process of eventually managing to extract the wanted information seems challenging. We could suppose though, that with the proper handling the results could prove rewarding for both the businesses and the customers. Based on this conjecture, we try to respond to this thesis how could a company of the tourism industry, using cluster analysis, take deal with its complex data and create value for itself and its customers.
\section{Research Methology}
\section{Thesis Outline}
The main goal of this thesis is to experiment with the use of clustering algorithms with categorical data and identify the value of its results in the formation of services, and the creation of marketing strategies in the tourism sector. Firstly, it defines machine learning and its categories, concentrating mainly on clustering and its applications in various sectors. Next, it analyses more in-depth two clustering methods, the Kmeans, and Agglomerative clustering. Further, in the last section of the research background, a thorough description of two clustering applications in the tourism sector are illustrated. Then, the case study description, available data, methodology, and results are presented. In conclusion, the results of the two algorithms are compared and their avail is discussed.